\documentclass[BTech]{iitpdiss}
\usepackage{times}
 \usepackage{t1enc}
\usepackage{url}
\usepackage{color}
\usepackage{graphicx} % for including graphics files
\usepackage{ifpdf} % to use same .tex file for both latex & pdflatex the following specifies different options to hyperref depending on whether latex or pdflatex is being run.
\ifpdf
\usepackage[colorlinks=,linkcolor=blue,urlcolor=blue,citecolor=blue,plainpages=false,pdfpagelabels,breaklinks]{hyperref}
\else
\usepackage[colorlinks,linkcolor=blue,urlcolor=blue,citecolor=blue,plainpages=false,pdfpagelabels,linktocpage]{hyperref}
\fi
\usepackage{epstopdf}
\usepackage{hyperref} % hyperlinks for references.
\usepackage{amsmath,amssymb} % easier math formulae, align, subequations \ldots

\begin{document}

%%%%%%%%%%%%%%%%%%%%%%%%%%%%%%%%%%%%%%%%%%%%%%%%%%%%%%%%%%%%%%%%%%%%%%
% Title page

\title{OPTIMAL CHARGING OF HYBRID ELECTRIC VEHICLE IN SMART GRID ENVIRONMENT}

\author{Venkatesh Chaturwedi}
\rollno{1101EE37}
\date{SEPTEMBER 2014}
\department{ELECTRICAL ENGINEERING}

%\nocite{*}
\maketitle

%%%%%%%%%%%%%%%%%%%%%%%%%%%%%%%%%%%%%%%%%%%%%%%%%%%%%%%%%%%%%%%%%%%%%%
% Certificate
\certificate

\vspace*{0.5in}

\noindent This is to certify that the thesis titled {\bf OPTIMAL CHARGING OF HYBRID ELECTRIC VEHICLE IN SMART GRID ENVIRONMENT}, submitted by {\bf Venkatesh Chaturwedi }, 
  to the Indian Institute of Technology, Patna, for
the award of the degree of {\bf Bachelor of Technology}, is a bona fide
record of the research work done by him under our supervision.  The
contents of this thesis, in full or in parts, have not been submitted
to any other Institute or University for the award of any degree or
diploma.

\vspace*{1.5in}

\begin{singlespacing}
\hspace*{-0.25in}
%\parbox{2.5in}{
%\noindent {\bf Dr.~} \\
%\noindent Research Guide \\ 
%\noindent Assistant Professor \\
%\noindent Dept. of Electrical Engineering\\
%\noindent IIT-Patna, 800 013 \\
%} 
\hfill 
\parbox{2.5in}{
\noindent {\bf Dr. S. Sivasubramani } \\
\noindent Supervisor \\ 
\noindent Assistant Professor \\
\noindent Dept.  of  Electrical Engineering\\
\noindent IIT-Patna, 800 013 \\
}  
\end{singlespacing}
\vspace*{0.25in}
\\
\noindent Place: Patna\\
\noindent Date: 17th September 2014 


%%%%%%%%%%%%%%%%%%%%%%%%%%%%%%%%%%%%%%%%%%%%%%%%%%%%%%%%%%%%%%%%%%%%%%
% Acknowledgements
\acknowledgements

Thanks to all those who made \TeX\ and \LaTeX\ what it is today.

%%%%%%%%%%%%%%%%%%%%%%%%%%%%%%%%%%%%%%%%%%%%%%%%%%%%%%%%%%%%%%%%%%%%%%
% Abstract

\abstract

\noindent KEYWORDS: \hspace*{0.5em} \parbox[t]{4.4in}{\LaTeX ; Thesis;
  Style files; Format.}

\vspace*{24pt}

\noindent A \LaTeX\ class along with a simple template thesis are
provided here.  These can be used to easily write a thesis suitable
for submission at IIT-Patna.  The class provides options to format
PhD, MS, M.Tech.\ and B.Tech.\ thesis.  It also allows one to write a
synopsis using the same class file.  Also provided is a BIB\TeX\ style
file that formats all bibliography entries as per the IITP format.

The formatting is as (as far as the author is aware) per the current
institute guidelines.

\pagebreak

%%%%%%%%%%%%%%%%%%%%%%%%%%%%%%%%%%%%%%%%%%%%%%%%%%%%%%%%%%%%%%%%%
% Table of contents etc.

\begin{singlespace}
\tableofcontents
\thispagestyle{empty}

\listoftables
\addcontentsline{toc}{chapter}{LIST OF TABLES}
\listoffigures
\addcontentsline{toc}{chapter}{LIST OF FIGURES}
\end{singlespace}


%%%%%%%%%%%%%%%%%%%%%%%%%%%%%%%%%%%%%%%%%%%%%%%%%%%%%%%%%%%%%%%%%%%%%%
% Abbreviations
\abbreviations

\noindent 
\begin{tabbing}
xxxxxxxxxxx \= xxxxxxxxxxxxxxxxxxxxxxxxxxxxxxxxxxxxxxxxxxxxxxxx \kill
\textbf{IITP}   \> Indian Institute of Technology, Patna \\
\textbf{RTFM} \> Read the Fine Manual \\
\end{tabbing}

\pagebreak

%%%%%%%%%%%%%%%%%%%%%%%%%%%%%%%%%%%%%%%%%%%%%%%%%%%%%%%%%%%%%%%%%%%%%%
% Notation

\chapter*{\centerline{NOTATION}}
\addcontentsline{toc}{chapter}{NOTATION}

\begin{singlespace}
\begin{tabbing}
xxxxxxxxxxx \= xxxxxxxxxxxxxxxxxxxxxxxxxxxxxxxxxxxxxxxxxxxxxxxx \kill
\textbf{$r$}  \> Radius, $m$ \\
\textbf{$\alpha$}  \> Angle of thesis in degrees \\
\textbf{$\beta$}   \> Flight path in degrees \\
\end{tabbing}
\end{singlespace}

\pagebreak
\clearpage

% The main text will follow from this point so set the page numbering
% to arabic from here on.
\pagenumbering{arabic}


%%%%%%%%%%%%%%%%%%%%%%%%%%%%%%%%%%%%%%%%%%%%%%%%%%
% Introduction.

\chapter{INTRODUCTION}
\label{chap:intro}

Ever since machine learning has been introduced into the field of computer science, it has been used in various fields. One of the most important application of machine learning is in the field of artificial neural networks. Artificial neural networks are used in various fields such as pattern recognition, image processing, speech recognition, etc. 

\section{Artifical Neural Networks}

\begin{singlespace}
\bibliographystyle{IEEEtran}
  \bibliography{refs}
\end{singlespace}


%%%%%%%%%%%%%%%%%%%%%%%%%%%%%%%%%%%%%%%%%%%%%%%%%%%%%%%%%%%%
% List of papers

%\listofpapers

%\begin{enumerate}  
%\item Authors....  \newblock
% Title...
%  \newblock {\em Journal}, Volume,
%  Page, (year).
%\end{enumerate}  
%
\end{document}
