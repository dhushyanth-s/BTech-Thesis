\chapter{Conclusion and Future Works}

\section{Conclusion}

PEMAN structure \cite{demarinisCodesignedIntegratedPhotonic2022} is a hybrid photonic electronic structure that emulates the functions of a singular neuron. In this thesis, the possibility to use a complete neural network using a single PEMAN structure is explored.

The concept using PEMAN to both train and infer a linear neural network is studied. Having an accuracy of 96.5\% versus 98.1\% of the non-quantized version, it was shown that even though there is an inevitable loss of accuracy, the PEMAN structure can be used to train and infer a neural network with a reasonable accuracy and good speeds.

CNN models using PEMAN structure was also studied. An analogy for CNN as a combination of linear neural network implemented as PEMAN was introduced and a model was trained as proof of concept, which had an accuracy of 98.80\%. This also showed that CNN are possible to be implemented by utilizing the PEMAN structure.

\section{Future Works}

The PEMAN structure is a very new concept and is still in its infancy. There are many possible ways to improve the structure and the performance of the structure. Some of the possible future works are listed below.

The structures main bottleneck right now is the analog section. Lacking a good way to apply non-linear functions in the photonic domain, we are forced to use an ADC. Future works could explore replace the ADC with a faster component which will improve the overall speed of the structure.

The random-convolution optimization is an interesting problem. The problem right now is that the accuracyis very low. This can be studied in more detail and understand the capabilities of the algorithm. If it gives acceptable accuracy, the number operations would reduce by a significant amount.